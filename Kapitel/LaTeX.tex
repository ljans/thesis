\chapter{\textrm\LaTeX}

\section{Textmodus}

Dieses Kapitel ist hauptsächlich dafür gedacht, die Ausgabe mit dem Quellcode zu vergleichen, um etwas Erfahrung im Textsatz mit \LaTeX\ zu gewinnen.
Makros -- die zentralen Steuerelemente -- bestehen aus einem \textbackslash\ gefolgt von einem Namen.
Die einfachste Gruppe von Makros ist für das Einfügen von Sonderzeichen gedacht, beispielsweise für Symbole wie \checkmark\ aber auch den \LaTeX-Schriftzug.

Andere Makros benötigen ein Argument, das nach dem Namen in geschweiften Klammern angegeben wird.
Dazu gehören beispielsweise die Makros \texttt{chapter} oder \texttt{section} zum Erstellen von Überschriften.
Auch Makros zur Formatierung von \textit{kursivem} oder \underline{unterstrichenem} Text funktionieren auf diese Weise.

\textbf{Achtung:} Besteht das Argument aus nur einem Zeichen, kann es ohne Klammern angegeben werden, zum Beispiel \underline so.
Das hat allerdings zur Folge, dass auch Leerzeichen nach einem Makro als Argument interpretiert werden.
So wird zwischen den Ausdrücken \LaTeX Test kein Leerzeichen gesetzt, obwohl es im Quellcode vorkommt.
Auch mehrere Leerzeichen in Folge helfen nicht, da diese    automatisch    reduziert    werden.
Zur Abhilfe kann hier mit einem Backslash gefolgt von einem Leerzeichen aber bewusst ein Leerzeichen \ \ \ \ gesetzt werden.
Anspruchsvollere Makros können auch mehrere Argumente erfordern, die dann nacheinander jeweils in geschweiften Klammern dem Namen angehängt werden.

Als Nächstes betrachtet man eine \emph{Umgebung}, die vom Makro \texttt{begin} geöffnet und von \texttt{end} geschlossen wird.
Darunter fällt zum Beispiel die Aufzählungs-Umgebung \emph{itemize}:

\begin{itemize}
	\item Erster Aufzählungspunkt
	\item Zweiter Aufzählungspunkt
	\item ...
\end{itemize}

Wurde im Präambel eine Literaturliste geladen, dann können im Text leicht Quellenangaben gemacht werden \parencite{Tonomura}.
Nach \textcite{Ohanian} stehen hierfür mehrere Makros zur Verfügung.
Diese lassen einen optionalen Parameter zu, der in eckigen Klammern angegeben wird beispielsweise eine Seitenzahl angibt.
(vgl. \cite[S. 7]{Tonomura})


\section{Mathematikmodus}

Standardmäßig befindet sich \LaTeX\ im \emph{Textmodus}, zum Setzen mathematischer Ausdrücke muss in den \emph{Mathematikmodus} umgeschaltet werden.
Dies ist zum Beispiel durch das Dollarzeichen \$ möglich.
In einem Fließtext lässt sich damit eine kleine Formel wie etwa $a = b + c$ einbinden.

Für größere Ausdrücke gibt es eigene Mathematik-Umgebungen, zum Beispiel \texttt{align}:
\begin{align} \label{eq:Erste_Formel}
	a = b + c.
\end{align}
Hierbei wird automatisch eine Formelnummer generiert, die über \eqref{eq:Erste_Formel} referenziert werden kann, falls ihr ein entsprechendes \texttt{label} zugewiesen wurde.
Die \emph{gesternte} Umgebung \texttt{align*} unterdrückt das Erzeugen einer Formelnummer.

Im Mathematikmodus gibt es nun eine Fülle weiterer Makros für den Aufbau der gewünschten Ausdrücke.
Das folgende Beispiel zeigt eine kleine Auswahl.
Es ist zu beachten, dass Makros allgemein auch verschachtelt genutzt werden können.
Das doppelte Backslash bewirkt einen Zeilenumbruch.

\begin{align*}
	f(x) = \alpha x + \beta \\
	\frac{1}{q} + \frac{1}{p} = n! \\
	\forall n \in \NN: \sqrt{5\pi^n} \geq 7 \\
	\mathcal F \vec x =: \lim_{b \to \infty} \int_a^b \langle \psi | \phi \rangle \delta_k(x) \, \mathrm dx.
\end{align*}

Sollen Formeln an einem bestimmten Zeichen (beispielsweise dem Gleichheitszeichen) ausgerichtet werden, kann das \&-Zeichen davorgesetzt werden.
Häufig verwendete Formelbausteine können über \texttt{newcommand} als eigene Makros definiert werden (siehe Präambel), um den Code lesbarer zu gestalten:
\begin{align*}
	\sumfty \abs{x_n + z_n}^2 &\leq \sumfty \Bracket{\abs{x_n + z_n}^2 + \abs{x_n - z_n}^2}\\
	&= \sumfty \bigBracket{\bracket{\con x_n + \con z_n}\bracket{x_n + z_n} + \bracket{\con x_n - \con z_n}\bracket{x_n - z_n}}\\
	&= \sumfty \Bracket{2\abs{x_n}^2 + 2\abs{z_n}^2}\\
	&= 2 \sumfty \abs{x_n}^2 + 2 \sumfty \abs{z_n}^2 < \infty.
\end{align*}
Kleinere Mathematik-Ausdrücke ohne Ausrichtung können etwas kompakter auch über die Umgebung mit den Begrenzern \texttt{\textbackslash[} und \texttt{\textbackslash]} gesetzt werden.

Für mathematische Sätze, Beweise, Definitionen, usw. werden ebenfalls entsprechende Umgebungen eingesetzt.
Auch diese lassen einen optionalen Parameter zu, der als Titel fungiert.

\begin{definition}[Hilbertraum]
	Ein Innenproduktraum, der bezüglich der vom Skalarprodukt induzierten Norm vollständig ist, heißt auch \emph{Hilbertraum}.
\end{definition}

\begin{theorem}[\textsc{Cauchy}-\textsc{Schwarz}'sche Ungleichung]
	Ist $X$ ein Skalarproduktraum, dann gilt für alle $x, z \in X$
	\[\bigAbs{\braket xz}^2 \leq \braket xx \braket zz.\]
\end{theorem}

\begin{proof}
	Seien $x,z \in X$ beliebig.
	Da die Behauptung für $z = 0$ trivialerweise erfüllt ist, nimmt man im Folgenden jedoch $z \neq 0$ an.
	Für alle $\alpha \in \CC$ gilt
	\[0 \leq \braket{x + \alpha z}{x + \alpha z} = \braket xx + \alpha \braket xz + \con\alpha \braket zx + \abs\alpha^2 \braket zz.\]
	Einsetzen von
	\[\alpha = -\frac{\braket zx}{\braket zz} = -\frac{\con{\braket xz}}{\braket zz}\]
	liefert
	\[0 \leq \braket xx - \frac{\abs{\braket xz}^2}{\braket zz} - \frac{\abs{\braket zx}^2}{\braket zz} + \frac{\abs{\braket zx}^2}{\braket zz} = \braket xx - \frac{\abs{\braket xz}^2}{\braket zz}\]
	und nach Multiplikation mit $\braket zz$
	\[0 \leq \braket xx \braket zz - \abs{\braket xz}^2 \text{,}\]
	woraus die Behauptung folgt.
\end{proof}
