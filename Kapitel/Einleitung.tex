\addchap{Einleitung}

Die geläufigsten Texteditoren (z.~B. Word) basieren auf dem Prinzip \enquote{what you see is what you get}, kurz WYSIWYG.
Dadurch ist deren Bedienung sehr intuitiv und leicht zu erlernen.
Für anspruchsvollere Dokumente -- wie etwa eine Abschlussarbeit -- können diese Editoren jedoch an ihre Grenzen stoßen.
Besonders im MINT-Bereich bietet sich stattdessen \LaTeX\ für den Textsatz an.
Dabei handelt es sich um eine Art Programmiersprache, in der Inhalte zunächst ohne Vorschau auf das eigentliche Dokument verfasst werden.
Um Gestaltungselemente wie etwa Überschriften hervorzuheben, werden diese nicht etwa ausgewählt und per Klick auf ein Menüelement formatiert, sondern durch sogenannte \emph{Makros} ausgezeichnet.
Dabei handelt es sich im Wesentlichen um Funktionen, die beispielsweise eine Veränderung der Schriftgröße ihres Arguments bewirken, bei dem es sich um den zu formatierenden Text handelt.
Die tatsächliche Gestalt des fertigen Dokuments ergibt sich erst durch \emph{Kompilieren} des entsprechenden Quellcodes zu einer PDF-Datei.

In dieser Einleitung sollen die Grundzüge des wissenschaftlichen Arbeitens mit \LaTeX\ nähergebracht sowie eine handliche Entwicklungsumgebung dafür bereitgestellt werden.
