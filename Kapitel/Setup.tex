\chapter{Setup}

\section{Installation}

\subsection{MiKTeX}
Um den \LaTeX-Quellcode kompilieren zu können, wird neben dem Texteditor ein weiteres Programm benötigt.
In dieser Einführung kommt dafür \enquote{MiKTeX} zum Einsatz, das unter \url{https://miktex.org/download} verfügbar ist.


\subsection{Visual Studio Code}
Als Editor für die \LaTeX-Dateien wird hier Visual Studio Code (kurz VS Code) verwendet, der unter \url{https://code.visualstudio.com/} erhältlich ist.
Nach der Installation können darüber die folgenden Erweiterungen (Extensions) bezogen werden:

\paragraph{\enquote{LaTeX Workshop} von James Yu.}
Diese Erweiterung stattet VS Code mit umfangreichen Hilfsmitteln rund um \LaTeX\ aus.

\paragraph{\enquote{LTeX -- LanguageTool grammar/spell checking} von Julian Valentin.}
Dieses Tool bietet eine beeindruckend gute Rechtschreibprüfung für \LaTeX-Dokumente.


\subsection{GIT}
Die flache Dateistruktur von \LaTeX\ eignet sich gut für eine Versionsverwaltung, hier kommt dafür GIT zum Einsatz, das unter \url{https://git-scm.com/download/win} verfügbar ist.
Bei der Installation werden zahlreiche Konfigurationsmöglichkeiten abgefragt, in den meisten Situationen kann allerdings der Standard verwendet werden.


\section{Das Repository}

Nun kann das Repository dieses Projekts von \url{https://github.com/ljans/thesis} geklont werden, am besten in ein Arbeitsverzeichnis wie etwa \texttt{C:\textbackslash Workspace\textbackslash}.
Hat alles geklappt, dann kann das Projekt jetzt in VS Code geöffnet werden.
Es enthält bereits einige Ordner und Dateien, deren Bedeutung im Folgenden erklärt wird.

\paragraph{.vscode} Dieser Ordner enthält Konfigurationsdateien für VS Code, in diesem Fall die sogenannten \enquote{Workspace-Settings}.
Speziell für dieses Projekt werden darin zahlreiche Einstellungen für die Erweiterung \enquote{LaTeX Workshop} getroffen, um ein reibungsloses Arbeiten mit \LaTeX\ zu ermöglichen.

\paragraph{Grafiken} In diesem Ordner werden Screenshots (oder andere Grafiken) abgelegt, die in diesem Dokument eingebunden werden.

\paragraph{Kapitel} Hier sind die Inhalte der jeweiligen Kapitel dieses Dokuments in eigenen Dateien ausgelagert.
Dadurch lässt sich die Größe der einzelnen Dateien reduzieren und somit die Übersichtlichkeit verbessern.

\paragraph{.gitattributes und .gitignore} Dies sind Konfigurationsdateien für die Versionsverwaltung GIT und können unberührt gelassen werden.

\paragraph{Literatur.bib} In diese Datei werden die (exemplarischen) Quellen der Arbeit in einer speziellen Syntax angelegt.
Im eigentlichen Dokument wird diese Literaturliste dann eingebunden, um die automatische Erstellung von Quellenangaben in \LaTeX\ nutzen zu können.
Der Name \enquote{Literatur} kann dabei angepasst werden und auch mehrere Dateien sind möglich.

\paragraph{Thesis.pdf} Dies ist die Ausgabedatei, die von MiKTeX aus dem \LaTeX-Quellcode generiert wird.

\paragraph{Thesis.tex} Hierbei handelt es sich um die \enquote{Hauptdatei} des Projekts.
In ihr befindet sich die sogenannte \emph{Präambel} mit den Metadaten des Dokuments; darin werden zum Beispiel Funktionspakete geladen, Einstellungen vorgenommen und Makros definiert.
Neben der Steuerung von Abbildungs-, Inhalts- und Literaturverzeichnis werden in dieser Datei dann auch die ausgelagerten Kapitel eingebunden.


\section{Kompilierung}

Ist eine \LaTeX-Datei in VS Code geöffnet, kann über die Erweiterung \enquote{LaTeX Workshop} im linken oberen Teil des Fensters ein manueller Kompiliervorgang gestartet werden (siehe \Cref{fig:Manuelles_Kompilieren}).
\begin{figure}[htbp]
	\centering
	\includegraphics[width=0.3\textwidth]{Grafiken/Manuelles_Kompilieren}
	\caption{Manuelles Kompilieren einer \LaTeX-Datei.}
	\label{fig:Manuelles_Kompilieren}
\end{figure}
Bei der Option \enquote{Full typesetting} wird der Vorgang mehrmals durchlaufen, was gerade beim ersten Mal notwendig ist, damit alle Querverweise und Referenzen richtig zugeordnet werden.
Das Projekt ist so konfiguriert, dass beim Speichern einer \LaTeX-Datei automatisch ein \enquote{Partial typesetting} ausgeführt wird, das zwar schneller geht, aber nur für Änderungen am Text verwendet werden sollte.

Beim ersten Kompilieren wird MiKTeX erkennen, dass die eingebundenen Pakete zum Teil noch nicht installiert sind und entsprechend dazu auffordern.
Dieser Vorgang kann einen Moment dauern und sich mehrmals wiederholen.
Anschließend startet die eigentliche Kompilierung.
Dabei werden einige Hilfsdokumente im Projekt angelegt, die ein erneutes Kompilieren beschleunigen sollen.

Nun kann das fertige PDF-Dokument angezeigt werden; dafür nutzt man die Schaltfläche im oberen rechten Teil des Fensters, wie in \Cref{fig:PDF_anzeigen} zu sehen.
\begin{figure}[htbp]
	\centering
	\includegraphics[width=0.4\textwidth]{Grafiken/PDF_anzeigen}
	\caption{Kompilierte PDF-Datei anzeigen.}
	\label{fig:PDF_anzeigen}
\end{figure}
Wird die Datei auf diesem Weg geöffnet, aktualisiert sich das PDF-Dokument bei jedem erneuten Kompilieren automatisch.
Darüber hinaus steht so \emph{SyncTeX} zur Verfügung:
Durch Linksklick auf eine Stelle im PDF-Dokument bei gehaltener \texttt{Strg}-Taste, springt VS Code automatisch an die jeweilige Stelle im Quellcode.
Andersherum wird die aktuelle Cursorposition in VS Code durch die Tastenkombination \texttt{Strg + Alt + J} im PDF-Dokument hervorgehoben.
